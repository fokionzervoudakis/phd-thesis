\chapter{Introduction}

The increasing complexity of \emph{uninhabited aerial vehicle} (UAV) missions, which may involve the execution of sophisticated (swarming) tactics by semi- or even fully-autonomous UAVs, is overwhelming mission development personnel. The development process is supported technologically by advanced, academic and commercially-oriented, mission planning systems and simulation environments. Orbit Logic's UAV Planner, for example, is a commercial simulation environment that provides resource, and task order, definition and management; manual and automated planning; and interactive visualizations~\cite{orbitlogic}. In the academic space, MissionLab from Georgia Tech is a tool for specifying and controlling multi-agent missions~\cite{gatech}. Nowak et al.\ have developed a system named SWARMFARE, which executes self-organizing swarm-based simulations to test search and destroy scenarios~\cite{Nowak_2007}. Multi-agent self-organization has also been used to model UAV swarms and thereby facilitate solutions to problems of multi-objective optimization~\cite{Perron_2008,Nowak_2008}.

Mission optimization constitutes a multi-objective problem because UAV missions are characterized by multiple and conflicting properties that form complex, and poorly understood, associations. The problem of multi-objective optimization has been studied extensively with evolutionary algorithms~\cite{Zitzler_2004}. With respect to UAV missions, evolutionary algorithms are used to form robust UAV-based air-to-ground communication networks~\cite{Agogino_2012}, and to generate a set of optimized and coordinated flight routes for surveillance missions~\cite{Stouch_2011}. The weight configurations that prioritize self-organizing rules in SWARMFARE are evolved with a genetic algorithm~\cite{Nowak_2007}. Rosenberg et al.\ use a simulation environment to evaluate optimized air campaign mission plans generated by a \emph{multi-objective evolutionary algorithm} (MOEA)~\cite{Rosenberg_2008}. Pohl and Lamont use a MOEA to optimize the concurrent routing of multiple UAVs to multiple targets~\cite{Pohl_2008}.

The problem of UAV scheduling/routing has also been modeled as a minimum cost network flow problem~\cite{Shetty_2008}; a traveling salesman problem; and a dynamic programming problem~\cite{Ahner_2006}. Alver et al.\ implement a route planning algorithm based on the time-oriented nearest neighbor heuristic~\cite{Alver_2012}. Kamrani and Ayani use a special-purpose simulation tool named S2-Simulator to generate flight routes for the surveillance of moving targets~\cite{Kamrani_2007}. In other simulation-related research, Corner and Lamont use a parallel discrete event simulation to explore emergence in UAV swarms~\cite{Corner_2004}. Lian and Deshmukh use \emph{Markov decision processes} (MDPs) to model UAVs in a dynamic multi-agent system where agents have to manage constraints, negotiate flight paths and avoid enemy positions in order to execute a set of goals~\cite{Lian_2006}. Hamilton et al.\ emphasize the importance of valid simulation models, and propose a testbed to empirically validate simulations against historical data~\cite{Hamilton_2007}.

When integrated, planning and simulation software should ideally provide information superiority and, ultimately, competitive advantage~\cite{Cebrowski_1998}. But these systems are currently challenged by the size and complexity of the UAV domain~\cite{Chasparis_2005,Jang_2005,Russo_2006}. The state space of any given mission is potentially enormous. Mission correctness is contingent on multiple factors including UAV flight trajectories, interoperability and conflict resolution capabilities; and a dynamic environment that encompasses variable terrain, unpredictable weather and moving targets. In this context, mission plans may contain errors that compromise mission correctness. We propose to support mission developers by analyzing UAV mission plans with software verification methods---and in particular, probabilistic model checking---that can detect mission-critical errors before real-world execution. To this end, the following thesis describes a novel method for domain-specific model checking called \emph{cascading verification}~\cite{Zervoudakis_2013}, and the application of that method to the probabilistic verification of complex UAV mission plans.

\section{Research Problem and Scope}
\label{sec:Research_Problem_and_Scope}

Model checking is an established formal verification method whereby a model checker systematically explores the state space of a system model to verify that each state satisfies a set of desired behavioral properties~\cite{Baier_2008}.

Research in model checking has focused on enhancing the efficiency and scalability of verification by employing partial order reduction, and by exploiting symmetries and other state space properties. This research is important because it mitigates the complexity of model checking algorithms, thereby enabling model builders to verify larger, more elaborate models. But the complexity associated with model and property specification has yet to be sufficiently addressed. Popular model checkers tend to support low-level modeling languages that require intricate models to represent even the simplest systems. For example, PROMELA, the modeling language used by the model checker Spin, is essentially a dialect of the relatively low-level programming language~C\@. Due to lack of appropriate control structures, the modeling language used by the probabilistic model checker PRISM forces model builders to pollute model components with variables that act as counters. These variables are manipulated at runtime to achieve desirable control flow from otherwise unordered commands.

Modeling complexity arises in part from the need to encode \emph{domain knowledge}---including domain objects and concepts, and their relationships---at relatively low levels of abstraction. We will demonstrate that, once formalized, domain knowledge can be reused to enhance the abstraction level of model and property specifications, and the effectiveness of probabilistic model checking.

Leveraged appropriately, formal domain knowledge can decrease specification and verification costs. On the verification side, the model checking framework Bogor achieves significant state space reductions in model checking of program code by exploiting characteristics of the program code's deployment platform~\cite{Robby_2003}. On the specification side, \emph{semantic model checking} supplements model checking with semantic reasoning over domain knowledge encoded in the \emph{Web Ontology Language} (OWL). Semantic model checking has been used to verify Web services~\cite{Narayanan_2002,Di_Pietro_2012}; Web service security requirements~\cite{Boaro_2010}; probabilistic Web services~\cite{Oghabi_2011}; Web service interaction protocols~\cite{Ankolekar_2005}; and Web service flow~\cite{Liu_2008}. Additionally, multi-agent model checking has been used to verify OWL-S process models~\cite{Lomuscio_2009}.

OWL is a powerful knowledge representation formalism, but expressive and reasoning limitations constrain its utility in the context of semantic model checking; for example, OWL cannot reason about triangular or self-referential relationships. As a W3C-approved OWL extension, the Semantic Web Rule Language (SWRL) addresses some of these limitations by integrating OWL with Horn-like rules. But OWL+SWRL cannot reason effectively with negation. The \emph{logic programming} (LP) language Prolog can be used to overcome problems that are intractable in OWL+SWRL\@. Prolog, however, lacks several of the expressive features afforded by OWL including support for equivalence and disjointness.

In summary, model checking is a prominent formal verification method. Contemporary modeling languages induce modeling complexity, which is exacerbated by the need to encode domain knowledge at relatively low levels of abstraction. Semantic model checking uses semantic reasoning over domain knowledge encoded in OWL to supplement model checking and thereby decrease specification costs. But expressive and reasoning limitations constrain OWL and, by extension, the potential of semantic model checking.

\section{Thesis Contributions}

This thesis presents four research contributions. First, we describe a novel model checking method that leverages domain knowledge to realize a non-trivial reduction in the effort required to specify system models and behavioral properties. The method uses a composite inference mechanism to reason about high-level system specifications and thereby synthesize low-level PRISM code for probabilistic model checking.

Second, we use domain-specific software development to structure formal verification for the UAV domain. Third, we implement a prototype system that exploits our method to verify mission plans. Fourth, we evaluate our prototype to demonstrate the utility of cascading verification in the context of a significant and novel application domain. The following sections elaborate these contributions.

\subsection{Cascading Verification}

We have designed a novel method for domain-specific model checking called cascading verification. Our method uses composite reasoning over high-level system specifications and formalized domain knowledge to synthesize \emph{both} low-level system models and the behavioral properties that need to be verified with respect to those models. In particular, model builders use a high-level \emph{domain-specific language} (DSL) to encode system specifications that can be analyzed with model checking. A \emph{compiler} uses automated reasoning to verify the consistency between each specification and domain knowledge encoded in OWL+SWRL and Prolog, which are combined to overcome their individual limitations. If consistency is deduced, then explicit and inferred domain knowledge is used by the compiler to synthesize a \emph{discrete-time Markov chain} (DTMC) model and \emph{probabilistic computation tree logic} (PCTL) properties from template code. PRISM subsequently verifies the model against the properties. Thus, verification \emph{cascades} through several stages of reasoning and analysis.

Our method gains significant functionality from each of its constituent technologies. OWL supports expressive knowledge representation and efficient reasoning; SWRL extends OWL with Horn-like rules that can model complex relational structures and self-referential relationships; Prolog extends OWL+SWRL with the ability to reason effectively with negation; DTMC introduces the ability to formalize probabilistic behavior; and PCTL supports the elegant expression of probabilistic properties.

Cascading verification is illustrated with a prototype system that verifies the correctness of UAV missions. We use the prototype to analyze~58 mission plans, which are based on real-world mission scenarios developed independently by the Defense Advanced Research Projects Agency (DARPA)~\cite{DARPA} and the Defence Research and Development Canada (DRDC) agency~\cite{Youngson_2004}. UAVs are contextualized by a particularly interesting and important experimental domain. The stochastic nature of UAV missions led us to select \emph{probabilistic model checking}, and in particular the popular tool PRISM~\cite{Hinton_2006}, for the verification of UAV mission plans.

As an implementation of cascading verification, our prototype realizes a non-trivial reduction in the effort required to specify system models and behavioral properties. For example, from 23 lines of YAML code comprising 92 tokens, cascading verification synthesizes 104 lines of PRISM code comprising 744 tokens and three behavioral properties (with our prototype, model builders encode mission specifications in a domain-specific dialect of the human-readable YAML format~\cite{Evans}).

\subsection{Domain-Specific Modeling for the UAV Domain}

When compared with general-purpose programming languages, DSLs provide increased expressivity and usability for software development in the context of specific application domains~\cite{Mernik_2005}. A DSL is defined by concrete syntax and an abstract syntax metamodel~\cite{Rivera_2009}. Abstract syntax specifies language concepts and their relationships, while concrete syntax specifies the notation that represents those concepts. \emph{Domain-specific modeling} (DSM) is a model-driven software development process that uses DSLs to encode system aspects. Syntax-oriented DSM avoids the behavioral properties of DSL models and metamodels. Several proposals attempt to specify these properties and thereby enable the verification of DSL-based systems with model checking and other formal analysis techniques.

OWL supports the development of domain-specific languages and systems by representing knowledge in a manner that is unambiguous for humans and computers~\cite{Musen_2004,Walter_2009}. For example, OWL can be used to formalize domain knowledge that constitutes the DSL metamodel~\cite{Walter_2009}. By encompassing OWL ontologies, cascading verification is a method with the facility to support (probabilistic) model checking for DSL-based systems. In this context, we exploit OWL to formalize a subset of the UAV domain. The resulting ontology underpins, and thereby constrains and structures, the concrete syntax of a YAML DSL\@. For each DSL-based mission specification, consistency between concrete syntax and the abstract syntax metamodel is enforced by a compiler prior to the synthesis of PRISM code.

Cascading verification encompasses a DSL to enhance the abstraction level of model and property specifications. Model builders use this DSL to encode system specifications for probabilistic model checking. If UAV mission specifications encoded with the YAML DSL are also scheduled for real-world execution (via some process that is outside the scope of our method), then our prototype implementation of cascading verification will in essence be supporting DSM for the UAV domain. This observation implies a link between mission and software development, thereby justifying to some extent our motivation to analyze complex missions with software verification methods.

\subsection{Prototype Design and Implementation}

To investigate the feasibility of cascading verification, we designed and implemented a prototype that uses our method to verify UAV missions. DSM structured the development of~58 mission plans that underpin both our research effort and the evaluation of our method and prototype. Mission plans are encoded with the concrete syntax of a bespoke YAML DSL, which was established for this project. The abstract syntax metamodel of that DSL forms part of a knowledge base that ultimately comprises semantic, rule-based and behavioral models, and a set of behavioral properties. Models and properties, which describe different aspects of the UAV domain, are encoded with OWL+SWRL, Prolog, and PRISM DTMC and PCTL templates. The reasoning methods supported by OWL, SWRL and Prolog are combined to form a composite inference mechanism that achieves OWL-LP integration.

\subsection{Prototype Evaluation}

With the evaluation of our prototype, we aim to demonstrate the utility of cascading verification in the context of a non-trivial application domain. To this end, the evaluation is primarily concerned with abstraction and effectiveness. Abstraction is analyzed by comparing the \emph{lines of code} (LOC) and numbers of lexical tokens required to specify UAV missions in YAML against the LOC and tokens that constitute synthesized PRISM code. We analyze effectiveness by presenting errors that can only be effectively eliminated with the automated synthesis of PRISM artifacts. We also consider the utility of the DTMC and PCTL artifacts synthesized by our prototype. Finally, we discuss the portability of cascading verification. While the evaluation presented in this thesis is preliminary, it is also a substantive analysis supported by two case studies (involving tactical and traffic surveillance missions) and~58 mission plans (involving mission characteristics borrowed from DARPA and DRDC).

\section{Thesis Outline}

The remainder of this thesis is structured as follows.

{
\parindent=0em
\parskip=\medskipamount

\newcommand{\myindent}[1]{\hangindent=2em\textsf{#1}}

\myindent{Chapter~\ref{chap:Background}} presents background material on the technologies---including OWL+SWRL, Prolog, DTMC and PCTL---that constitute cascading verification. This chapter also provides an overview of the UAV domain and complex UAV missions.

\myindent{Chapter~\ref{chap:Method_Overview}} presents a high-level overview of cascading verification. This chapter also specifies Mission~A, an example UAV mission that underpins the discussion in subsequent chapters.

\myindent{Chapter~\ref{chap:Domain_Modeling}} describes how domain knowledge can be encoded in OWL+SWRL, Prolog, and DTMC and PCTL templates. The application of these technologies is illustrated with respect to a running example (Mission~A) and two case studies.

\myindent{Chapter~\ref{chap:Cascading_Verification}} describes our prototype implementation of cascading verification for the UAV domain by tracing verification from high-level system specifications, which are encoded in a domain-specific YAML dialect, to probabilistic model checking with PRISM\@. This chapter also presents the technologies and implemented components that constitute our prototype.

\myindent{Chapter~\ref{chap:Evaluation}} evaluates the benefits afforded by our prototype, and the utility and portability of cascading verification.

\myindent{Chapter~\ref{chap:Conclusions_and_Future_Work}} summarizes our contributions to semantic model checking and discusses directions for future work.

\myindent{Appendix~\ref{chap:Threat_Area_Calculations}} presents a framework of equations for establishing the occurrence, and calculating the duration, of \emph{threat area incursions} committed by UAVs.

\myindent{Appendix~\ref{chap:Ontology}} contains an ontology encoded in OWL+SWRL that formalizes a subset of the UAV domain. This ontology encompasses both generic mission concepts and specialized knowledge related to tactical and traffic surveillance missions.

\myindent{Appendix~\ref{chap:Prolog_Knowledge_Base}} contains Prolog rules that augment the ontological domain knowledge encoded in OWL+SWRL\@.

\myindent{Appendix~\ref{chap:PRISM_Templates}} contains DTMC and PCTL templates encoded in the programming language Ruby that formalize probabilistic behavioral knowledge.

\myindent{Appendix~\ref{chap:DSL_Schema}} contains a schema definition for the YAML DSL presented in this thesis.

\myindent{Appendix~\ref{chap:Mission_Verification_Artifacts}} contains verification artifacts---including system specifications encoded in YAML, synthesized DTMC and PCTL artifacts, and PRISM output---for representative UAV mission plans.

}

\noindent Source code for the software presented in this thesis is available online at: \url{https://github.com/fokionzervoudakis/mission-verification-framework}
